%%%%%%%%%%%%%%%%%%%%%%%%%%%%%%%%%%%%%%%%%
% Journal Article
% LaTeX Template
% Version 1.4 (15/5/16)
%
% This template has been downloaded from:
% http://www.LaTeXTemplates.com
%
% Original author:
% Frits Wenneker (http://www.howtotex.com) with extensive modifications by
% Vel (vel@LaTeXTemplates.com)
%
% License:
% CC BY-NC-SA 3.0 (http://creativecommons.org/licenses/by-nc-sa/3.0/)
%
%%%%%%%%%%%%%%%%%%%%%%%%%%%%%%%%%%%%%%%%%

%----------------------------------------------------------------------------------------
%	PACKAGES AND OTHER DOCUMENT CONFIGURATIONS
%----------------------------------------------------------------------------------------

\documentclass[twoside,twocolumn,paper=letter]{article}
\usepackage[cm]{fullpage} 

\usepackage{blindtext} % Package to generate dummy text throughout this template 

\usepackage[sc]{mathpazo} % Use the Palatino font
\usepackage[T1]{fontenc} % Use 8-bit encoding that has 256 glyphs
\linespread{1.05} % Line spacing - Palatino needs more space between lines
\usepackage[protrusion=true,expansion=true]{microtype} % Slightly tweak font spacing for aesthetics

\usepackage[english]{babel} % Language hyphenation and typographical rules

%\usepackage[hmarginratio=1:1,top=32mm,columnsep=20pt]{geometry} % Document margins
\usepackage[hang, small,labelfont=bf,up,textfont=it,up]{caption} % Custom captions under/above floats in tables or figures
\usepackage{booktabs} % Horizontal rules in tables

\usepackage{lettrine} % The lettrine is the first enlarged letter at the beginning of the text

\usepackage{enumitem} % Customized lists
\setlist[itemize]{noitemsep} % Make itemize lists more compact

\usepackage{abstract} % Allows abstract customization
\renewcommand{\abstractnamefont}{\normalfont\bfseries} % Set the "Abstract" text to bold
\renewcommand{\abstracttextfont}{\normalfont\small\itshape} % Set the abstract itself to small italic text

\usepackage{titlesec} % Allows customization of titles
\renewcommand\thesection{\Roman{section}} % Roman numerals for the sections
\renewcommand\thesubsection{\roman{subsection}} % roman numerals for subsections
\titleformat{\section}[block]{\large\scshape\centering}{\thesection.}{1em}{} % Change the look of the section titles
\titleformat{\subsection}[block]{\large}{\thesubsection.}{1em}{} % Change the look of the section titles

\usepackage{fancyhdr} % Headers and footers
\pagestyle{fancy} % All pages have headers and footers
\fancyhead{} % Blank out the default header
\fancyfoot{} % Blank out the default footer
%\fancyhead[C]{Running title $\bullet$ May 2016 $\bullet$ Vol. XXI, No. 1} % Custom header text
\fancyfoot[RO,LE]{\thepage} % Custom footer text

\usepackage{titling} % Customizing the title section

\usepackage{hyperref} % For hyperlinks in the PDF

%----------------------------------------------------------------------------------------
%	TITLE SECTION
%----------------------------------------------------------------------------------------

\setlength{\droptitle}{-4\baselineskip} % Move the title up

\pretitle{\begin{center}\Large\bfseries} % Article title formatting
\posttitle{\end{center}} % Article title closing formatting
\title{Applying Machine Learning to Predict and Explain Primate Consortship} % Article title
\author{%
\textsc{Josh King} \\[1ex] 
\and 
\textsc{Vayu Kishore} \\[1ex] 
\and 
\textsc{Filippo Ranalli} \\[1ex] 
}
\date{} % Leave empty to omit a date
\renewcommand{\maketitlehookd}{%
%\begin{abstract}
%\noindent \blindtext % Dummy abstract text - replace \blindtext with your abstract text
%\end{abstract}
}

%----------------------------------------------------------------------------------------

\begin{document}

% Print the title
\maketitle

%----------------------------------------------------------------------------------------
%	ARTICLE CONTENTS
%----------------------------------------------------------------------------------------

\section{Introduction}

\lettrine[nindent=0em,lines=3]{L} orem ipsum dolor sit amet, consectetur adipiscing elit.
% describe prior work
\blindtext % Dummy text

%------------------------------------------------

\section{Dataset}
% describe source
% describe raw features
% describe transformed features
% describe special characteristics (imbalanced labels)
\section{Methods}
\subsection{Classification}
\subsection{Clustering}
\subsection{Graphical Approach}

%------------------------------------------------

\section{Results}
\subsection{Classification}
\subsection{Clustering}
\subsection{Graphical Approach}

%------------------------------------------------

\section{Discussion}
\section{Next Steps}

%----------------------------------------------------------------------------------------
%	REFERENCE LIST
%----------------------------------------------------------------------------------------

\begin{thebibliography}{99} % Bibliography - this is intentionally simple in this template

\bibitem[Tung et al., 2012]{Tung:2012}
  Jenny Tung, Marie J. E. Charpentier, Sayan Mukherjee, Jeanne Altmann, and Susan C. Alberts (2012)
\newblock 
  Genetic Effects on Mating Success and Partner Choice in a Social Mammal.
\newblock {\em The American Naturalist}, 2012 180:1, 113--129.
 
\end{thebibliography}

%----------------------------------------------------------------------------------------

\end{document}
